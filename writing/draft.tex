\documentclass[10pt,a4paper,titlepage,twoside]{article}
\usepackage[utf8]{inputenc}
\usepackage[T1]{fontenc}
\usepackage[english]{babel}


\usepackage{amsmath, amssymb, amsfonts, amsthm, mathtools}
% mathtools for: Aboxed (put box on last equation in align envirenment)
\usepackage{microtype} %improves the spacing between words and letters

\usepackage{graphicx}
%\graphicspath{ {./pics/} {./eps/}}
\usepackage{epsfig}
\usepackage{epstopdf}
\usepackage{ulem}
\usepackage{outlines}

\newcommand\given[1][]{\:#1\vert\:}


%%%%%%%%%%%%%%%%%%%%%%%%%%%%%%%%%%%%%%%%%%%%%%%%%%
%% COLOR DEFINITIONS
%%%%%%%%%%%%%%%%%%%%%%%%%%%%%%%%%%%%%%%%%%%%%%%%%%
\usepackage[svgnames]{xcolor} % Enabling mixing colors and color's call by 'svgnames'
%%%%%%%%%%%%%%%%%%%%%%%%%%%%%%%%%%%%%%%%%%%%%%%%%%
\definecolor{MyColor1}{rgb}{0.2,0.4,0.6} %mix personal color
\definecolor{newred}{HTML}{C5000B} %mix personal color

\newcommand{\textb}{\color{Black} \usefont{T1}{lmss}{m}{n}}
\newcommand{\blue}{\color{MyColor1} \usefont{T1}{lmss}{m}{n}}
\newcommand{\blueb}{\color{MyColor1} \usefont{T1}{lmss}{b}{n}}
\newcommand{\red}{\color{newred} \usefont{T1}{lmss}{m}{n}}
\newcommand{\green}{\color{Turquoise} \usefont{T1}{lmss}{m}{n}}
\newcommand{\bluen}{\color{MyColor1} \usefont{T1}{phv}{b}{n}}
%%%%%%%%%%%%%%%%%%%%%%%%%%%%%%%%%%%%%%%%%%%%%%%%%%

\usepackage[norule,hang,flushmargin]{footmisc}
\usepackage[font=footnotesize, hang]{caption}


%%%%%%%%%%%%%%%%%%%%%%%%%%%%%%%%%%%%%%%%%%%%%%%%%%
%% FONTS AND COLORS
%%%%%%%%%%%%%%%%%%%%%%%%%%%%%%%%%%%%%%%%%%%%%%%%%%
%    SECTIONS
%%%%%%%%%%%%%%%%%%%%%%%%%%%%%%%%%%%%%%%%%%%%%%%%%%
\usepackage{titlesec}
\usepackage{sectsty}
%%%%%%%%%%%%%%%%%%%%%%%%
%set section/subsections HEADINGS font and color
\sectionfont{\color{MyColor1}}  % sets colour of sections
\subsectionfont{\color{MyColor1}}  % sets colour of sections

%set section enumerator to arabic number (see footnotes markings alternatives)
\renewcommand\thesection{\arabic{section}.} %define sections numbering
\renewcommand\thesubsection{\thesection\arabic{subsection}} %subsec.num.


%%%%%%%%%%%%%%%%%%%%%%%%%%%%%%%%%%%%%%%%%%%%%%%%%%
%		CAPTIONS
%%%%%%%%%%%%%%%%%%%%%%%%%%%%%%%%%%%%%%%%%%%%%%%%%%
\usepackage{caption}
\usepackage{subcaption}
%%%%%%%%%%%%%%%%%%%%%%%%
\captionsetup[figure]{labelfont={color=MyColor1}}

%%%%%%%%%%%%%%%%%%%%%%%%%%%%%%%%%%%%%%%%%%%%%%%%%%
%		!!!EQUATION (ARRAY) --> USING ALIGN INSTEAD
%%%%%%%%%%%%%%%%%%%%%%%%%%%%%%%%%%%%%%%%%%%%%%%%%%
%using amsmath package to redefine eq. numeration (1.1, 1.2, ...) 
%%%%%%%%%%%%%%%%%%%%%%%%
\renewcommand{\theequation}{\thesection\arabic{equation}}

%set box background to grey in align environment 
\usepackage{etoolbox}% http://ctan.org/pkg/etoolbox
\makeatletter
\patchcmd{\@Aboxed}{\boxed{#1#2}}{\colorbox{black!15}{$#1#2$}}{}{}%
\patchcmd{\@boxed}{\boxed{#1#2}}{\colorbox{black!15}{$#1#2$}}{}{}%
\makeatother
%%%%%%%%%%%%%%%%%%%%%%%%%%%%%%%%%%%%%%%%%%%%%%%%%%

\renewcommand{\rmdefault}{phv} % arial
\renewcommand{\sfdefault}{phv} % arial


% Page layout (geometry)
%\setlength\voffset{-1in}
%\setlength\hoffset{-1in}
%\setlength\topmargin{2.5cm}
%\setlength\rightmargin{1.5cm}
%\setlength\oddsidemargin{2.5cm}
%\setlength\textheight{22.4cm}
%\setlength\textwidth{14.00cm}
%\setlength\footskip{1.3cm}
%\setlength\headheight{0cm}
%\setlength\headsep{0cm}

\usepackage
[
        a4paper,% other options: a3paper, a5paper, etc
        left=2.5cm,
        right=2.5cm,
        top=3cm,
        bottom=3cm,
]
{geometry}


% Paragraph formatin
\linespread{1}
\setlength{\parindent}{0pt}
\setlength{\parskip}{6pt plus 3pt minus 3pt}


%%%%%%%%%%%%%%%%%%%%%%%%%%%%%%%%%%%%%%%%%%%%%%%%%%
%		Import listings package and define parameters
%%%%%%%%%%%%%%%%%%%%%%%%%%%%%%%%%%%%%%%%%%%%%%%%%%
\usepackage{listings}
\definecolor{grey}{rgb}{0.9,0.9,0.9}
\definecolor{mygreen}{rgb}{0,0.6,0}
\lstset{backgroundcolor=\color{grey},frame=single, language=Matlab, basicstyle=\tiny,commentstyle=\color{mygreen},keywordstyle=\color{blue}
}

% Yma colors
\definecolor{ymagray}{rgb}{0.2,0.2,0.2}
\definecolor{ymablue}{RGB}{0,132,209}
\definecolor{ymadblue}{RGB}{0,69,134}
\definecolor{ymagreen}{HTML}{579d1c}
\definecolor{ymaorng}{HTML}{FF950E}

\definecolor{ymared}{HTML}{c5000b}


%
\usepackage{float}
\usepackage{caption}
\usepackage{subcaption}

% - - - - - - - - - - Fancy Header - - - - - - - - - - %
%% L/C/R denote left/center/right header (or footer) elements
%% E/O denote even/odd pages
\usepackage{fancyhdr}
\pagestyle{fancy}
\fancyhead[LO,RE]{\slshape\thepage}
\renewcommand{\headrulewidth}{0.1pt}
\cfoot{}


\makeatletter
\let\reftagform@=\tagform@
\def\tagform@#1{\maketag@@@{(\ignorespaces\textcolor{ymadblue}{#1}\unskip\@@italiccorr)}}
\renewcommand{\eqref}[1]{\textup{\reftagform@{\ref{#1}}}}
\makeatother
\usepackage{hyperref}
\hypersetup{colorlinks=true,linkcolor={ymadblue}}


\newcommand{\tilv}[1]{\textbf{\color{ymagreen}[#1]}}
\newcommand{\note}[1]{\textbf{[\color{ymablue}#1}]}
\newcommand{\highlight}[1]{{\bluen{#1}}}
\newcommand{\hh}[1]{\textbf{\color{ymaorng} #1}}
\newcommand{\point}[1]{\textbf{\color{ymared} #1}}
\newcommand{\rem}[1]{{\red{\textbf{REMOVE:}} \red{\sout{#1}}}}

\usepackage[onehalfspacing]{setspace}


%%%%%%%%%%%%%%%%%%%%%%%%%%%%%%%%%%%%%%%%%%%%%%%%%%
%% Hyphenation correction
%%%%%%%%%%%%%%%%%%%%%%%%%%%%%%%%%%%%%%%%%%%%%%%%%%
%\babelhyphenation[icelandic]{
%  Forsendurnar
%  líkinda
%}


%%%%%%%%%%%%%%%%%%%%%%%%%%%%%%%%%%%%%%%%%%%%%%%%%%
%% PREPARE TITLE
%%%%%%%%%%%%%%%%%%%%%%%%%%%%%%%%%%%%%%%%%%%%%%%%%%
\title{\blue Thesis - draft \\
\blueb NB-IoT}
\author{Sigurgeir Gunnarsson \\KTH - Communication Systems}
\date{\today}
%%%%%%%%%%%%%%%%%%%%%%%%%%%%%%%%%%%%%%%%%%%%%%%%%%


\begin{document}
\maketitle


\thispagestyle{empty}
\tableofcontents

%\cleardoublepage
\newpage
\setcounter{page}{1}



% -=-=-=-=-=-=-=-=-=-=-=-=-=-=-=-=-=-=-=-=-=-=-=-=-=-=-=-=-=-=-=-=-=-=-=-=-=-=-
% Inngangur
% -=-=-=-=-=-=-=-=-=-=-=-=-=-=-=-=-=-=-=-=-=-=-=-=-=-=-=-=-=-=-=-=-=-=-=-=-=-=-
\section{Intro}

%\hglght{* Describe the problem at hand}

The idea of having all devices connected has led to the concept of Internet of Things (IoT). Here all devices such as electric or water meters, household devices and about every thing with built in sensors, are communicating with each other or with a central server via short distance communication or wide area networks (WAN).

One implementation is utilising the LTE standard and modifying it to meet the requirement needed for IoT. Besides the massive volumes it has to handle, there are requirements on coverage, communication complexity and battery life \cite{wang}.

The devices need to be accessible from hard radio conditions like from house cellar or a technique room in the core. The communications have to be simple for the sake of resilience and low power consumption. Lastly it is expected that the battery to last the live of the device. It is therefore extremely important that no unnecessary transmission is undertaken.

When determining the capacity of the base station, or the serving cell, it is not enough to have the theoretical capacity analysed. The random nature of when the device transmits, increases the chance of collision when the number of devices increases. As mentioned the requirement is to avoid such collision. The capacity therefore reduces when taken this into account.

A suitable approach is to identify the theoretical capacity of an NB-IoT cell. Then introducing randomness into the access, taking into account the possibility of collision. The result is a multi-dimensional space of capacity weighted against probability of collision and number of devices. To further enrich the analysis, the radio condition can be taken into account, introducing further randomness since poorer radio condition results in higher re-transmission probability when the number of devices increases.

% Making this happening brings out a challenge due to the amount of devices, connected to a single cell in the cellular network. All these devices extract resources, they interfere with each other and their location is probably deep in the house or in the cellar, where communication is hard to withheld.

%The standard has taken into account the attenuation in the radio, adding robustness in the transmission providing up to 20 dB higher pathloss. When this is combined with the need for long battery life, the reliability of the transmission becomes important. If the battery limit is regarded as a finite pool of transmissions, it becomes unnecessary usage to re-transmit. 

\vspace{2cm}

\point{The challenge is on the physical channels, when the load increases. The random access can result in collision, making it necessary to resend the package. Re-transmission causes unnecessary communication and extra battery. This is therefore something to avoid.}

\point{Keep in mind that NB-IoT is UL-driven service, unlike traditional mobile services that are DL-driven}

% -=-=-=-=-=-=-=-=-=-=-=-=-=-=-=-=-=-=-=-=-=-=-=-=-=-=-=-=-=-=-=-=-=-=-=-=-=-=-
% Kafli
% -=-=-=-=-=-=-=-=-=-=-=-=-=-=-=-=-=-=-=-=-=-=-=-=-=-=-=-=-=-=-=-=-=-=-=-=-=-=-
\clearpage
\section{NB-IoT}\label{theo}

3GPP defines wide area network (WAN) narrowband internet-of-things (NB-IoT) to handle communication with connected devices. The physical channel structure is the same as LTE, but scaled down to occupy less bandwidth. The downscaling provides flexibility to place the carrier where there is available space. One solution is to place it on a single  GSM transmission channel (TCH), and that way integrate it with the GSM network. The other two are co-existing with LTE, where one uses the guard band between LTE channels and the other is embedded into the LTE carrier \cite{schlienz}.

The high throughput characteristics of LTE, and the fact that NB-IoT devices do not need to send big volumes of data, gives an opportunity to increase the protection of the transmitted data. This is done by extra padding, more robust coding scheme and modulation. The benefit is increased link budget, allowing up to 20 dB extra attenuation, and decreased probability of wrongly received data, requiring retransmission \cite{wang}.

NB-IoT devices differ from other mobile devices, in that they aren't downlink critical. The sensor deployment requires data to be transmitted from the sensors, towards other devices or a central service. The uplink is therefore the important direction. It can be further assumed that the devices initiates the communication, not the base station.

An additional design criteria for NB-IoT, is long battery life \cite{wang}. Having the extra link budget, it can transmitt with lower power but it is also important to avoid unnecessary retransmission. In the case of retransmission, there is waiting - awake - time and extra transmission. Both require extra battery usage which has to be avoided.

%The LTE guard band varies in width, depending on carrier bandwidth. Most deployments of LTE though, use 5 MHz or larger bandwidth, since the smaller bandwidths introduce exceptions from the normal structure.

%The sub-carrier definition for the in-band case, is the same as used by the LTE control channels. The NB-IoT transmission can therefore be carried alongside normal LTE traffic if there is capacity to share the control space \point{Need to confirm if it is correctly understood that LTE and NB-IoT share the control space and if they can co-exist on the same channel}.

%NB-IoT is defined based on the LTE carrier structure. The smallest LTE transmission element is resource element (RE) which uses 15 kHz bandwidth. A resource block (RB) is 12 RE and occupies 180 kHz (12*15 kHz=180kHz). The LTE carrier can be of varying bandwidth, from 1,4 MHz to 20 MHz, where the difference lies in the number of RB used.

%Since the structure is common, NB-IoT can be defined as functionality within the LTE carrier. According to Schlienz (2106) \cite{schlienz} there are specific RB indexes where NB-IoT can be activated. That way both functionalities (LTE and NB-IoT) are active at the same time.

\subsection{Channel structure}

NB-IoT downlink is defined to use 12 sub-carriers, with each sub-carrier having bandwidth of 15 kHz. That adds up to 180 kHz, which fits perfectly into the GSM transmission channel (TCH) of 200 kHz. The uplink uses either a single sub-carrier, which can be of varying bandwidth: 3,75 kHz or 15 kHz, or it uses 3, 6, or 12 sub-carriers of 15 kHz bandwidth each\cite{3gpp}. The last is the same as for the downlink and the total bandwidth becomes 180 kHz.


\hh{It is important to describe the physical channels, since the analysis goes on about identifying when they reach congestion}. \point{Needs to be investigated if the higher layer channels introduce any limitation.}

\subsection{Access}

As mentioned, NB-IoT is uplink critical communication meaning that the device initiates the communication and the data exchanged is something the device has been collecting. This means that one of the most critical capacity aspects of NB-IoT is how many devices can be accessing the network and how long does it take for the device to ,,get rid of the data''.

The access process is as follows: \hh{These steps have to be verified from \cite{3gpp}}
\begin{enumerate}
  \item Random access: the device initiates a connection with the base station using the random channel
  \begin{enumerate}
    \item Here the possibility of collision with other devices occurs i.e. when two devices try to access at same time they interfere and neither of them gets through
    \item \hh{What is the retry mechanism ?}
  \end{enumerate}
  \item The base station assigns resources on the dedicated channels \point{freeing resources from the random access channel ??}
  \item When the device has gotten its dedicated resources it can begin transmitting
\end{enumerate}

The capacity limit is reached when the number of devices increases, partly due to the probability of collisions and partly due to the capacity of the dedicated channels. \hh{The point of the paper is to identify which one is congested before.}

\point{Notes:}\vspace{-4mm}
\begin{itemize}
  \item \hh{Investigate how the paging mechanism is implemented in NB-IoT. There must be some way for the base station to reach the UE.}
\end{itemize}



\subsection{Evaluation points}

The limitations when it comes to serving multiple users lies within the random access procedure. The user has limited power and has therefore to make its transmission based on most efficient transmission methods. Collision is important to avoid. Modulation and coding scheme have to navigate between giving the maximum throughput but with best recovery of distorted bits \point{this should be looked into - find articles evaluating this trade-off}.

When the user has gotten the random access procedure completed and is scheduled on the transmission channels. The next step is the focus of the thesis - how to maximise ??, to be able to service more users and withholding the goal of minimum transmission.


\subsection{Extras}

Schlienz (2016) \cite{schlienz} is a very good summary of NB-IoT structure. The references are all 3GPP TS36.2xx, which is the channel structure of LTE - with a chapter covering NB-IoT.

Question: is it sufficient to rewrite Shclienz (2016) and refer to the article, plus the 3GGP standard ?

\point{* Describe NB-IoT}

There are three implementation methods of NB-IoT. There is standalone, guardband and in-band.

\textbf{6.4.1 Standalone Deployment:}
Standalone deployment is a deployment scenario in which operators deploy NB-IoT using
existing idle spectrum resources. These resources can be the operator’s spectrum
fragments with non-standard bandwidths or spared from other radio access technologies
(RATs) by refarming.

\textbf{6.4.2 LTE Guardband Deployment:}
Guardband deployment is a deployment scenario in which operators deploy NB-IoT in guard
bands within existing LTE spectrum resources.

\textbf{6.4.3 LTE In-band Deployment:}
In-band deployment is a deployment scenario in which operators deploy NB-IoT using
existing LTE in-band resource blocks (RBs).
\tilv{GSMA deployment guide}


In the 3GPP NB-IoT technical specification, chapter 4.2, the radio resources are defined in deep details (if relevant).
\cite{schlienz}.

In the following work of the focus will be on in-band LTE NB-IoT.

\note{WHERE IS TTI DESCRIBED !!!}

\point{* Cover the access method}

Schlienz (2016) is a really god reference to work the access method from.

* Coverage classes - how different classes affect the access mechanism

Here are some outstanding questions. One coverage class something mentioned in the specifikation or is it something suggested in literature.

% -=-=-=-=-=-=-=-=-=-=-=-=-=-=-=-=-=-=-=-=-=-=-=-=-=-=-=-=-=-=-=-=-=-=-=-=-=-=-
% HERMUN
% -=-=-=-=-=-=-=-=-=-=-=-=-=-=-=-=-=-=-=-=-=-=-=-=-=-=-=-=-=-=-=-=-=-=-=-=-=-=-
\clearpage
\section{Simulation}

To simulate the environment there are a number of key features that need to be taken into account:

\begin{itemize}
\item What is the expected number of UE trying to send \note{base this on articles}
\item How frequent do the UE send like eg. how many times a day \note{base this on articles}
\item What is the package size that is sent per transmission (full session) \note{base this on articles}
\item And how long does it take to process the package
\end{itemize}

The simulation is a Python script relying on the package SimPy. This is an event driven simulation package, but currently the events aren't used. Instead it is utilised such that one step represents one TTI. The script could be extended to have a limited number of RACH channels \note{assuming it is one available per TTI}, thus making use of the built in features as events.

\subsection{Assumptions}

The initial assumption, is to look at an area of one base station (eNodeB). According to Ericsson blog \cite{lme_web1} the base station can serve an area up to 40 km and have around 50.000 connections on the cell. A more pragmatic solution is to look at a rural area of 20 km radius. Such an area can have \hh{xx} homes. If a home is expected to have \hh{yy} NB-IoT devices, the scenario gives \hh{xx*yy} connections.

Here the focus is on smart metering devices. A measurement is sent every 5-15 minuttes and the size of the package is \hh{bb}. \note{The package size is random variable with \?\? distribution}




% -=-=-=-=-=-=-=-=-=-=-=-=-=-=-=-=-=-=-=-=-=-=-=-=-=-=-=-=-=-=-=-=-=-=-=-=-=-=-
% Heimildaskrá
% -=-=-=-=-=-=-=-=-=-=-=-=-=-=-=-=-=-=-=-=-=-=-=-=-=-=-=-=-=-=-=-=-=-=-=-=-=-=-
\clearpage
\section{References}

\begin{thebibliography}{4}
\bibitem{schlienz}
Schlienz, J., \& Raddino, D. (2016).
\textit{Narrowband internet of things whitepaper}. White Paper, Rohde\&Schwarz, 1-42.

\bibitem{wang}
Wang, Y. P. E., Lin, X., Adhikary, A., Grovlen, A., Sui, Y., Blankenship, Y., ... \& Razaghi, H. S. (2017).
\textit{A primer on 3GPP narrowband Internet of Things}. IEEE communications magazine, 55(3), 117-123.

\bibitem{3gpp}
3GPP TS 36.201,
\textit{3rd Generation Partnership Project; Technical Specification Group Radio Access Network; Evolved Universal Terrestrial Radio Access (E-UTRA); LTE physical layer; General description}, version 15.2.0, Release 15.

\bibitem{3gpp211}
3GPP TS 36.211,
\textit{3rd Generation Partnership Project; Technical Specification Group Radio Access Network; Evolved Universal Terrestrial Radio Access (E-UTRA); LTE physical layer; General description}, version 15.14.0, Release 15.

\bibitem{chen}
Chen, M., Miao, Y., Hao, Y., \& Hwang, K. (2017).
\textit{Narrow band internet of things}
. IEEE access, 5, 20557-20577.

\bibitem{lme_web1}
Ericsson. (2019).
\textit{What is NB-IoT? Practical tips to unlock its business potential}.
Ericsson blog.
\url{https://www.ericsson.com/en/blog/2019/10/what-is-nb-iot}

\end{thebibliography}


% -=-=-=-=-=-=-=-=-=-=-=-=-=-=-=-=-=-=-=-=-=-=-=-=-=-=-=-=-=-=-=-=-=-=-=-=-=-=-
% Glósur !!!!
% -=-=-=-=-=-=-=-=-=-=-=-=-=-=-=-=-=-=-=-=-=-=-=-=-=-=-=-=-=-=-=-=-=-=-=-=-=-=-
\clearpage
\section{\point{Outstanding}}

\begin{itemize}
\item Is there a code available that makes it possible to emulate the access mechanism and show the clash when the load increases ?

\end{itemize}

\hh{Statement:} For a device with 164 dB coupling loss, a 10-year battery life can be reached if the UE transmits 200-byte data a day on average {[TR 45.820 v13.1.0, “Cellular system support for ultra low complexity and low throughput internet of things,” Nov. 2015]}


\end{document}
